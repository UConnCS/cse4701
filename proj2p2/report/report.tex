\documentclass{article}
\usepackage[utf8]{inputenc}

\title{\textbf{CSE4701}\\ Project 2, Part 2}
\author{Mike Medved}
\date{November 24th, 2023}

\usepackage{graphicx}
\usepackage{amsthm}
\usepackage{amssymb} 
\usepackage{amsmath}
\usepackage{caption}
\usepackage{listings}
\usepackage{multirow, tabularx}
\usepackage[margin=1in]{geometry} 
\usepackage[table,xcdraw]{xcolor}
\usepackage{enumitem}
\newlist{parlist}{enumerate}{1}
\setlist[parlist]{label=(\alph*),wide=0pt,topsep=0pt}

% changes title name for the table of contents
\renewcommand*\contentsname{Table of Contents}

% makes sections unnumbered and hides numbers from table of contents
\setcounter{secnumdepth}{0}

\newcolumntype{C}{>{\centering\arraybackslash}X}
\NewExpandableDocumentCommand\mcc{O{1}m}{\multicolumn{#1}{c}{#2}}

\definecolor{codegreen}{rgb}{0,0.6,0}
\definecolor{codegray}{rgb}{0.5,0.5,0.5}
\definecolor{codepurple}{HTML}{C42043}
\definecolor{backcolour}{HTML}{F2F2F2}
\definecolor{bookColor}{cmyk}{0,0,0,0.90}  
\color{bookColor}
\input{js.tex}

\newcommand\numberstyle[1]{%
    \footnotesize
    \color{codegray}%
    \ttfamily
    \ifnum#1<10 0\fi#1 |%
}

\def\ojoin{\setbox0=\hbox{$\bowtie$}%
  \rule[-.02ex]{.25em}{.4pt}\llap{\rule[\ht0]{.25em}{.4pt}}}
\def\leftouterjoin{\mathbin{\ojoin\mkern-5.8mu\bowtie}}
\def\rightouterjoin{\mathbin{\bowtie\mkern-5.8mu\ojoin}}
\def\fullouterjoin{\mathbin{\ojoin\mkern-5.8mu\bowtie\mkern-5.8mu\ojoin}}

\begin{document}

\maketitle

\tableofcontents

\newpage

\section{Introduction}

This assignment reuses some of the logic for generating data from the previous assignment. As you can see below, the code is the same from last assignment.
$\hfill \break$
\lstinputlisting[language=JavaScript]{snippets/1-generate.ts}

$\hfill \break$
In order to initialize the SQL database, we will use the \textit{mysqldump} command to dump the state of the database as of Project 1 Part 1, and then reimport it into this database instance. Note that this is required because each assignment's databases are containerized via Docker and thus do not otherwise overlap. The commands used are shown below.

\vspace{0.25cm}
\begin{lstlisting}[language=bash]
# dump from project 1 part 1 database
mysqldump -h 127.0.0.1 -u root -p Book_Loan_DB > proj1p2.sql

# reimport into project 2 part 2 database
mysql -h 127.0.0.1 -u root -p Book_Loan_DB < ./proj1p2.sql
\end{lstlisting}

\newpage
\section{Question 1, Cross-Fetching Book Data}

We will be using the \textit{mariadb} npm package to interface with the MySQL database. Omitted from the below code snippet is setting up the connection pool upon which we can get a connection to perform queries. The code below highlights the query we perform as well the MongoDB \textit{findUnique} call to retrieve the image data for the associated book ID.

$\hfill \break$
Once the data from each database is retrieved and paired together, we iterate over all the books that have images and write to an image file by creating a buffer with the base64 encoded image data, then writing it to the filesystem.
$\hfill \break$

\lstinputlisting[language=JavaScript]{snippets/2-cross-fetch.ts}

$\hfill \break$
After executing the above code, a \textit{covers} directory is created with two image files, \textit{B3.jpg} and \textit{B5.jpg}. The images are shown below.

\begin{figure}[h!]
    \begin{minipage}[b]{0.5\linewidth}
      \centering
      \includegraphics[width=0.35\textwidth]{images/B3.jpg}
      \caption*{B3.jpg}
    \end{minipage}
    \begin{minipage}[b]{0.5\linewidth}
      \centering
      \includegraphics[width=0.35\textwidth]{images/B5.jpg}
      \caption*{B5.jpg}
    \end{minipage}
\end{figure}

\section{Question 2, Book Investment by Branch}

In order to compute the total investment for each branch's paper books, we will need to query the Book\_Copies table, which will give us the Book IDs, corresponding branches, and number of copies per branch.

$\hfill \break$
Next, we will query all of the books from the MongoDB database and get the paperback price for each book that has that data. Lastly, we will get the investment per-book per-branch and sum it up to get the final branch totals.

\vspace{0.25cm}
\lstinputlisting[language=JavaScript]{snippets/3-branch-investment.ts}

$\hfill \break$
Below, you can see the output of running the above script - note that the column names in the below table were generated by the built-in JavaScript \textit{console.table} function and are not indicative of how the data is stored or anything of the sort.

\begin{figure}[h!]
    \centering
    \includegraphics[scale=0.65]{images/branch-totals.png}
    \caption*{Branch Investment}
    \label{fig:branch-investment}
\end{figure}

\end{document}