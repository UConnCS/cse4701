\documentclass{article}
\usepackage[utf8]{inputenc}

\title{CSE 4701 — Homework 1}
\author{Mike Medved}
\date{September 14th, 2023}

\usepackage{graphicx}
\usepackage{amsthm}
\usepackage{amssymb} 
\usepackage{amsmath}
\usepackage{caption}
\usepackage{listings}
\usepackage{multirow, tabularx}
\usepackage[margin=1in]{geometry} 
\usepackage[table,xcdraw]{xcolor}
\usepackage{enumitem}
\newlist{parlist}{enumerate}{1}
\setlist[parlist]{label=(\alph*),wide=0pt,topsep=0pt}

\newcolumntype{C}{>{\centering\arraybackslash}X}
\NewExpandableDocumentCommand\mcc{O{1}m}{\multicolumn{#1}{c}{#2}}

\lstdefinestyle{Java}{
    language     = Java,
    aboveskip    = 3mm,
    belowskip    = 3mm,
    basicstyle   = \footnotesize\ttfamily,
    keywordstyle = \color{blue},
    stringstyle  = \color{green},
    commentstyle = \color{red}\small\ttfamily
}

\def\ojoin{\setbox0=\hbox{$\bowtie$}%
  \rule[-.02ex]{.25em}{.4pt}\llap{\rule[\ht0]{.25em}{.4pt}}}
\def\leftouterjoin{\mathbin{\ojoin\mkern-5.8mu\bowtie}}
\def\rightouterjoin{\mathbin{\bowtie\mkern-5.8mu\ojoin}}
\def\fullouterjoin{\mathbin{\ojoin\mkern-5.8mu\bowtie\mkern-5.8mu\ojoin}}

\begin{document}

\maketitle

\section*{8.16 Company Database}

\begin{parlist}
    \item Retrieve the names of all employees in department 5 who work more than 10 hours per week on the ProductX project.
    \begin{align*}
        \text{rel}_1 &\leftarrow \sigma_{\text{Pname} = \text{`ProductX'}}(\text{PROJECT}) \\
        \text{rel}_2 &\leftarrow \text{rel}_1 \bowtie \text{WORKS\_ON} \\
        \text{rel}_3 &\leftarrow \text{EMPLOYEE} *_{\text{SSN}=\text{Essn}}(\sigma_{\text{Hours} > 10}(\text{rel}_2)) \\
        \text{result} &\leftarrow \pi_{\text{Fname,Lname}}(\sigma_{\text{Dno}=5}(\text{rel}_3))
    \end{align*}

    \begin{table}[h!]
        \centering
        \caption*{Table 1: Operation result for 8.16(a)}
        \begin{tabularx}{0.5\textwidth}{|C|C|}
            \hline \textbf{Fname} & \textbf{Lname} \\ \hline
            John & Smith \\ \hline
            Joyce & English \\ \hline
        \end{tabularx}
    \end{table}

    \item List the names of all employees who have a dependent with the same first name as themselves.
    \begin{align*}
        \text{rel} &\leftarrow \text{EMPLOYEE} \bowtie_{\text{SSN=Essn AND Fname=DEPENDENT\_NAME}}(\text{DEPENDENT}) \\
        \text{result} &\leftarrow \pi_{\text{Fname,Lname}}(\text{rel})
    \end{align*}

    \begin{table}[h!]
        \centering
        \caption*{Table 2: Operation result for 8.16(b)}
        \begin{tabularx}{0.5\textwidth}{|C|C|}
            \hline \textbf{Fname} & \textbf{Lname} \\ \hline
             &  \\ \hline
        \end{tabularx}
    \end{table}

    \item Find the names of all employees who are directly supervised by `Franklin Wong'.
    \begin{align*}
        \text{rel}_1 &\leftarrow \sigma_{\text{Fname = Franklin AND Lname = Wong}}(\text{EMPLOYEE}) \\
        \text{rel}_2 &\leftarrow \text{EMPLOYEE} \bowtie_{\text{Super\_ssn=SSN}}(\text{rel}_1) \\
        \text{result} &\leftarrow \pi_{\text{Fname,Lname}}(\text{rel}_2)
    \end{align*}

    \begin{table}[h!]
        \centering
        \caption*{Table 3: Operation result for 8.16(c)}
        \begin{tabularx}{0.5\textwidth}{|C|C|}
            \hline \textbf{Fname} & \textbf{Lname} \\ \hline
            John & Smith \\ \hline
            Joyce & English \\ \hline
            Ramesh & Narayan \\ \hline
        \end{tabularx}
    \end{table}

    \item For each project, list the project name and the total hours per week (by all employees) spent on that project.
    \begin{align*}
        \text{rel}_1\left(\text{Pno}, \text{Total\_hours}\right) &\leftarrow \left[\text{Pno}, \sum \text{WORKS\_ON}\right] \\
        \text{rel}_2 &\leftarrow \text{rel}_1 \bowtie_{\text{Pno=Pnumber}}(\text{PROJECT}) \\
        \text{result} &\leftarrow \pi_{\text{Pname,Total\_hours}}(\text{rel}_2)
    \end{align*}

    \begin{table}[h!]
        \centering
        \caption*{Table 4: Operation result for 8.16(d)}
        \begin{tabularx}{0.5\textwidth}{|C|C|}
            \hline \textbf{Pname} & \textbf{Total\_hours} \\ \hline
            ProductX & 52.5 \\ \hline
            ProductY & 37.5 \\ \hline
            ProductZ & 50.0 \\ \hline
            Computerization & 55.0 \\ \hline
            Reorganization & 25.0 \\ \hline
            NewBenefits & 55.0 \\ \hline
        \end{tabularx}
    \end{table}

    \item Retrieve the names of all employees who work on every project.
    \begin{align*}
        \text{rel}_1(\text{Pno, Ssn}) &\leftarrow \pi_{\text{Pno, Essn}}(\text{WORKS\_ON}) \\
        \text{rel}_2 &\leftarrow \pi_{\text{Pnumber}}(\text{PROJECT}) \\
        \text{rel}_3 &\leftarrow \pi_{\text{Fname, Lname}}(\text{rel}_1 \div \text{rel}_2) \\
        \text{result} &\leftarrow \pi_{\text{Fname, Lname}}(\text{EMPLOYEE} * \text{rel}_3)
    \end{align*}

    \begin{table}[h!]
        \centering
        \caption*{Table 5: Operation result for 8.16(e)}
        \begin{tabularx}{0.5\textwidth}{|C|C|}
            \hline \textbf{Fname} & \textbf{Lname} \\ \hline
             &  \\ \hline
        \end{tabularx}
    \end{table}

    \item Retrieve the names of all employees who do not work on any project.
    \begin{align*}
        \text{rel}_1(\text{Pno, Ssn}) &\leftarrow \pi_{\text{Pno, Essn}}(\text{WORKS\_ON}) \\
        \text{rel}_2 &\leftarrow \pi_{\text{Pnumber}}(\text{PROJECT}) \\
        \text{rel}_3 &\leftarrow \pi_{\text{Fname, Lname}}(\text{rel}_1 - \text{rel}_2) \\
        \text{result} &\leftarrow \pi_{\text{Fname, Lname}}(\text{EMPLOYEE} * \text{rel}_3)
    \end{align*}

    \begin{table}[h!]
        \centering
        \caption*{Table 6: Operation result for 8.16(f)}
        \begin{tabularx}{0.5\textwidth}{|C|C|}
            \hline \textbf{Fname} & \textbf{Lname} \\ \hline
             &  \\ \hline
        \end{tabularx}
    \end{table}

    \newpage
    \item For each department, retrieve the department name and the average salary of all employees working in that department.
    \begin{align*}
        \text{rel}_1(\text{Dnumber, Avg\_salary}) &\leftarrow \left[\text{Dnumber}, \text{Avg}(\text{SALARY})\right] \\
        \text{result} &\leftarrow \pi_{\text{Dname, Avg\_salary}}(\text{rel}_1 * \text{DEPARTMENT})
    \end{align*}

    \begin{table}[h!]
        \centering
        \caption*{Table 7: Operation result for 8.16(g)}
        \begin{tabularx}{0.5\textwidth}{|C|C|}
            \hline \textbf{Dname} & \textbf{Avg\_salary} \\ \hline
            Research & 33250 \\ \hline
            Administration & 31000 \\ \hline
            Headquarters & 55000 \\ \hline
        \end{tabularx}
    \end{table}

    \item Retrieve the average salary of all female employees.
    \begin{align*}
        \text{result}(\text{Avg\_salary}) &\leftarrow \text{Avg}(\sigma_{\text{Sex = F}}(\text{EMPLOYEE})) \\
    \end{align*}

    \begin{table}[h!]
        \centering
        \caption*{Table 8: Operation result for 8.16(h)}
        \begin{tabularx}{0.5\textwidth}{|C|}
            \hline \textbf{Avg\_salary} \\ \hline
            31000 \\ \hline
        \end{tabularx}
    \end{table}
\end{parlist}

\newpage
\section*{8.18 Library Database}

\begin{parlist}
    \item How many copies of the book titled \textit{The Lost Tribe} are owned by the library branch whose name is `Sharpstown'?
    \begin{align*}
        \text{rel}_1 &\leftarrow \sigma_{\text{Branch\_name = Sharpstown}}(\text{LIBRARY\_BRANCH}) \\
        \text{rel}_2 &\leftarrow \sigma_{\text{Title = The Lost Tribe}}(\text{BOOK}) \\
        \text{result} &\leftarrow \pi_{\text{No\_of\_copies}}((A \times B) * (\text{BOOK\_COPIES})) 
    \end{align*}
    
    \item How many copies of the book titled \textit{The Lost Tribe} are owned by each library branch?
    \begin{align*}
        \text{rel}_1 &\leftarrow \sigma_{\text{Title = The Lost Tribe}}(\text{BOOK}) \\
        \text{rel}_2 &\leftarrow ((\text{rel}_1 \times \text{LIBRARY\_BRANCH}) * \text{BOOK\_COPIES}) \\
        \text{result} &\leftarrow \pi_{\text{Title, Branch\_name, No\_of\_copies}}(\text{rel}_2)
    \end{align*}
    
    \item Retrieve the names of all borrowers who do not have any books checked out.
    \begin{align*}
        \text{rel}_1 &\leftarrow \text{BORROWER} - (\text{BORROWER} * \text{BOOK\_LOANS}) \\
        \text{result} &\leftarrow \pi_{\text{Name}}(\text{rel}_1)
    \end{align*}
    
    \item For each book that is loaned out from the Sharpstown branch and whose Due\_date is today, retrieve the book title, the borrower's name, and the borrower's address.
    \begin{align*}
        \text{rel}_1 &\leftarrow \sigma_{\text{Branch\_name = Sharpstown}}(\text{LIBRARY\_BRANCH}) \\
        \text{rel}_2 &\leftarrow \sigma_{\text{Due\_date = TODAY()}}(\text{BOOK\_LOANS}) \\
        \text{rel}_3 &\leftarrow (\text{rel}_1 \times BOOK) * \text{rel}_2 \\
        \text{result} &\leftarrow \pi_{\text{Title, Name, Address}}(\text{rel}_3 * \text{BORROWER})
    \end{align*}
    
    \item For each library branch, retrieve the branch name and the total number of books loaned out from that branch.
    \begin{align*}
        \text{rel}_1(\text{Branch\_no, Total}) &\leftarrow \left[\text{Branch\_no}, \text{Count}(\text{Book\_id})\right](\text{BOOK\_LOANS}) \\ 
        \text{rel}_2 &\leftarrow \text{rel}_1 * \text{LIBRARY\_BRANCH} \\
        \text{result} &\leftarrow \pi_{\text{Branch\_name, Total}}(\text{rel}_2)
    \end{align*}
    
    \item Retrieve the names, addresses, and number of books checked out for all borrowers who have more than five books checked out.
    \begin{align*}
        \text{rel}_1(\text{Card\_no, Total}) &\leftarrow \left[\text{Card\_no}, \text{Count}(\text{Book\_id})\right](\text{BOOK\_LOANS}) \\
        \text{rel}_2 &\leftarrow \sigma_{\text{Total} > 5}(\text{rel}_1) \\
        \text{rel}_3 &\leftarrow \text{rel}_2 * \text{BORROWER} \\
        \text{result} &\leftarrow \pi_{\text{Name, Address, Total}}(\text{rel}_3)
    \end{align*}
    
    \item For each book authored (or coauthored) by \textit{Stephen King}, retrieve the title and the number of copies owned by the library branch whose name is `Central'.
    \begin{align*}
        \text{rel}_1 &\leftarrow \sigma_{\text{Branch\_name = Central}}(\text{LIBRARY\_BRANCH}) \\
        \text{rel}_2 &\leftarrow \sigma_{\text{Author\_name = Stephen King}}(\text{BOOK\_AUTHORS}) \\
        \text{rel}_3 &\leftarrow \text{rel}_2 * \text{BOOK} \\
        \text{rel}_4 &\leftarrow \text{rel}_1 * \text{BOOK\_COPIES} \\
        \text{result} &\leftarrow \pi_{\text{Title, No\_of\_copies}}(\text{rel}_3 * \text{rel}_4)
    \end{align*}

\end{parlist}

\newpage
\section*{8.22 Join Operations}

\begin{parlist}
    \item $T_1 \bowtie_{T_1.P = T_2.A} T_2$
    
    \begin{table}[h!]
        \centering
        \begin{tabularx}{0.5\textwidth}{|C|C|C|C|C|C|}
            \hline \textbf{P} & \textbf{Q} & \textbf{R} & \textbf{A} & \textbf{B} & \textbf{C} \\ \hline
            10 & a & 5 & 10 & b & 6 \\ \hline
            10 & a & 5 & 10 & b & 5 \\ \hline
            25 & a & 6 & 25 & c & 3 \\ \hline
        \end{tabularx}
    \end{table}
    
    \item $T_1 \bowtie_{T_1.Q = T_2.B} T_2$
    
    \begin{table}[h!]
        \centering
        \begin{tabularx}{0.5\textwidth}{|C|C|C|C|C|C|}
            \hline \textbf{P} & \textbf{Q} & \textbf{R} & \textbf{A} & \textbf{B} & \textbf{C} \\ \hline
            15 & b & 8 & 10 & b & 6 \\ \hline
            15 & b & 8 & 10 & b & 5 \\ \hline
        \end{tabularx}
    \end{table}
    
    \item $T_1 \leftouterjoin_{T_1.P = T_2.A} T_2$
    
    \begin{table}[h!]
        \centering
        \begin{tabularx}{0.5\textwidth}{|C|C|C|C|C|C|}
            \hline \textbf{P} & \textbf{Q} & \textbf{R} & \textbf{A} & \textbf{B} & \textbf{C} \\ \hline
            10 & a & 5 & 10 & b & 6 \\ \hline
            10 & a & 5 & 10 & b & 5 \\ \hline
            15 & a & 8 & \textit{null} & \textit{null} & \textit{null} \\ \hline
            25 & a & 8 & 25 & c & 3 \\ \hline
        \end{tabularx}
    \end{table}

    \item $T_1 \rightouterjoin_{T_1.Q = T_2.B} T_2$
    
    \begin{table}[h!]
        \centering
        \begin{tabularx}{0.5\textwidth}{|C|C|C|C|C|C|}
            \hline \textbf{P} & \textbf{Q} & \textbf{R} & \textbf{A} & \textbf{B} & \textbf{C} \\ \hline
            15 & b & 8 & 10 & b & 6 \\ \hline
            \textit{null} & \textit{null} & \textit{null} & 25 & c & 3 \\ \hline
            15 & b & 8 & 10 & b & 5 \\ \hline
        \end{tabularx}
    \end{table}
    
    
    \item $T_1 \cup T_2$
    
    \begin{table}[h!]
        \centering
        \begin{tabularx}{0.5\textwidth}{|C|C|C|}
            \hline \textbf{P} & \textbf{Q} & \textbf{R} \\ \hline
            10 & a & 5 \\ \hline
            15 & b & 8 \\ \hline
            25 & a & 6 \\ \hline
            10 & b & 6 \\ \hline
            25 & c & 3 \\ \hline
            10 & b & 5 \\ \hline
        \end{tabularx}
    \end{table}
    
    \item $T_1 \bowtie_{T_1.P = T_2.A \textbf{ AND } T_1.R = T_2.C} T_2$
    
    \begin{table}[h!]
        \centering
        \begin{tabularx}{0.5\textwidth}{|C|C|C|C|C|C|}
            \hline \textbf{P} & \textbf{Q} & \textbf{R} & \textbf{A} & \textbf{B} & \textbf{C} \\ \hline
            10 & a & 5 & 10 & b & 5 \\ \hline
        \end{tabularx}
    \end{table}

\end{parlist}

\end{document}